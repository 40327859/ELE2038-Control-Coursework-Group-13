
\section{Linearisation} 
Linearisation is used to approximate a linear system from a non-linear system. This linear system can be
found at the system’s equilibrium point. This allows us to analyse the system’s behaviour around the set point
and it is easier to analyse a linear system rather than a non-linear system.
To linearise, we use \texttt{sympy} to determine the partial derivatives of the system dynamics with respect to the states and the input. We define
\begin{equation}
    \phi = \frac{5}{7m}  \left(mg\sin{\phi} + c\frac{x_3^{e2}}{(\delta - x_1^{e})^2} -k(x_1^{e}-d) -bx_2^{e}   \right)
\end{equation}
\subsection*{Partial Derivatives of Phi}
\hfill \\
When we partially derive each equation, we get:
\begin{align}
\left.\frac{\partial \phi}{\partial x_1}\right|_{x_1^e, x_2^e, x_3^e,v^e} ={}& \frac{5 \left(\frac{2 c x_{3}^{2}}{\left(\delta - x_{1}\right)^{3}} - k\right)}{7 m}
\\
\left.\frac{\partial \phi}{\partial x_2}\right|_{x_1^e, x_2^e, x_3^e,v^e} ={}&  - \frac{5 b}{7 m}
\\
\left.\frac{\partial \phi}{\partial x_3}\right|_{x_1^e, x_2^e, x_3^e,v^e} ={}&  \frac{10 c x_{3}}{7 m \left(\delta - x_{1}\right)^{2}}
\\
\end{align}
\subsection*{Partial Derivatives of Psi} \hfill \\
\begin{equation}
    \psi = \frac{v - x_3R}{L_0 + L_1e^{-\alpha (\delta - x_1)^2}}
\end{equation}
When we partially derive each equation, we get:
\begin{align}
\left.\frac{\partial \psi}{\partial x_1}\right|_{x_1^e, x_2^e, x_3^e,v^e} ={}& - \frac{L_{1} \alpha \left(- R x_{3}^e + v\right) e^{- \alpha \left(\delta - x_{1}^e\right)}}{\left(L_{0} + L_{1} e^{- \alpha \left(\delta - x_{1}^e\right)}\right)^{2}}
\\
\left.\frac{\partial \psi}{\partial x_3}\right|_{x_1^e, x_2^e, x_3^e,v^e} ={}& - \frac{R}{L_{0} + L_{1} e^{- \alpha \left(\delta - x_{1}^e\right)}}
\\
\left.\frac{\partial \psi}{\partial v}\right|_{x_1^e, x_2^e, x_3^e,v^e} ={}& \frac{1}{L_{0} + L_{1} e^{- \alpha \left(\delta - x_{1}^e\right)}}
\\
\end{align}
\subsection*{Linearised Equations
}\hfill \\
Thus $\phi $ is:
\begin{equation}
    \phi \approx \underbrace{\frac{5 \left(\frac{2 c x_{3}^{e2}}{\left(\delta - x_{1}^e\right)^{3}} - k\right)}{7 m}}_A(x_1 - {x^e}_1) + \underbrace{\frac{-5 b}{7 m}}_B(x_2 - {x^e}_2) +  \underbrace{\frac{10 c x_{3}^e}{7 m \left(\delta - x_{1}^e\right)^{2}}}_C (x_3 - {x^e}_3)
\end{equation}
Thus $\psi $ is:
\\
\begin{equation}
    \psi \approx \underbrace{ \frac{-L_{1} \alpha \left(- R x_{3}^e + v\right) e^{- \alpha \left(\delta - x_{1}^e\right)}}{\left(L_{0} + L_{1} e^{- \alpha \left(\delta - x_{1}^e\right)}\right)^{2}}}_D(x_1 - {x^e}_1) + \underbrace{ \frac{-R}{L_{0} + L_{1} e^{- \alpha \left(\delta - x_{1}^e\right)}}}_E(x_2 - {x^e}_2) + \underbrace{\frac{1}{L_{0} + L_{1} e^{- \alpha \left(\delta - x_{1}^e\right)}}}_F (x_3 - {x^e}_3)
\end{equation}
\\
Then if we take $(x - x^e) = \Bar{x} $ and rewrite the equations in terms of $\Bar{x} $ we get:
\begin{align}
    \Bar{\Dot{x}}_1 &= \Bar{x}_2
    \\
    \Bar{\Dot{x}}_2 &= A\Bar{x}_1 + 
    B\Bar{x}_2 + C\Bar{x}_3
    \\
    \Bar{\Dot{x}}_3 &= D\Bar{x}_1 + 
    E\Bar{x}_3 + F\Bar{v}
\end{align}
\section{Transfer Function}